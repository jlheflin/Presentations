\documentclass{article}
\usepackage[letterpaper]{geometry}
\title{Developing Preorganized Chelators for Separating Nuclear Fission Products}
\author{Jacob Heflin}
\date{\today}

\begin{document}
\maketitle
v1: \\
The concentrations of nuclear fission products are statistically related to the identity of the source nuclear fuel.

Being able to selectively chelate these products as opposed to their contaminant counterparts helps to increase the
precision of the concentration measurement, leading to more accurate predictions of the
source material.

Typical contaminants to nuclear fission products are those that have the same properties when being
characterized by the same analytical methods.

Developing preorganized chelators for these species allows for their
selective chelation.

Preorganizing chelators involves studying and understanding current chelators and other functional
groups that bind well to the ions of interest.

This work focuses on the use of 3,4,3-LI(1,2-HOPO) as a chelator for
nuclear fission products, and exploring possible derivatives generated from HostDesigner when replacing the 4 carbon
linker.

These generated structures are then evaluated using a soon-to-be-published prediction model from the Critical
Materials Institude at Ames Lab, which is able to predict stability constants for the structures using their respective
SMILEs format.

These structures are then evaluated using DFT to ensure that the trends that are seen for the predictive
model are retained in the ab initio perspective.

v2: \\
The concentrations of nuclear fission products are statistically linked to the identity of their source nuclear fuel.
Accurate quantification of these products, however, is often hindered by the presence of contaminants with similar
chemical properties that complicate standard analytical methods. Selective chelation of fission products offers a
pathway to improving the precision of source material identification. This work explores the development of preorganized
chelators designed for selective binding of key fission products over their contaminants. Specifically, we focus on
modifying the well-established chelator 3,4,3-LI(1,2-HOPO) by varying its four-carbon linker using HostDesigner to
generate potential derivatives. These candidate structures are evaluated using a predictive stability constant model
developed by the Critical Materials Institute at Ames Laboratory, which estimates binding strengths from molecular
SMILES representations. Promising candidates are further validated using density functional theory (DFT) calculations
to assess whether predicted binding trends are maintained at the ab initio level. Post-analysis investigates structural
features contributing most significantly to selectivity enhancement, guiding future chelator design.
\end{document}
